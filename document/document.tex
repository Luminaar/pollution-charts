\documentclass[12pt]{article}

\usepackage[a4paper, lmargin=1in, rmargin=1in, tmargin=1in, bmargin=1in]{geometry}
\usepackage[czech]{babel}
\usepackage{fontspec}
\usepackage{hyperref}
\setmainfont{Latin Modern Roman}
\setmonofont{Latin Modern Mono}
\parindent=0pt
\parskip=1em
\usepackage{setspace}
\onehalfspacing

\usepackage[framemethod=tikz]{mdframed}
\mdfdefinestyle{anecdote}{%
	leftmargin=0.5cm,rightmargin=0.5cm,%
	innerleftmargin=1cm,innerrightmargin=1cm,innerbottommargin=1cm,%
	roundcorner=5pt,%
	backgroundcolor=purple!5,%
	align=center,%
	frametitleaboveskip=0.5cm%
}

\title{Vizualizace dat z Integravaného registru znečišťování}
\author{Bc. Maxim Kovykov}

\begin{document}
	
\begin{titlepage}
	\maketitle
	\thispagestyle{empty}
\end{titlepage}

\newpage


\section*{Úvod}
Cílem této práce je vytvořit vizualizace dat s využitím propojených dat, popsat poznatky spojené s tvorbou tohoto typu aplikace, zhodnotit výhody a nedostatky zvolených přístupů a okomentovat vhodnost technologií pro široké využívání.

Tato semestrální práce nemá textový ani věcný překryv, ani jinou věcnou souvislost, s jinými semestrálními nebo kvalifikačními pracemi, které autor zpracovával.


\section*{Integrovaný registr znečišťování}
Jako subjekt byla vybrána data z Integrovaného Registru Znečišťování (IRZ). IRZ je veřejný informační systém spravovaný Ministerstvem životního prostředí. Povinnost provozovat tento registr vyplývá z tzv. Aarhauské úmluvy o přístupu k informacím o znečišťování. Existence registu je dána zákonem č. 76/2002 Sb. (o integrované prevenci a omezování znečištění).

IRZ sleduje celkem 72 látek a rozlišuje různé druhy úniků (do ovzduší, půdy a vody) a navíc sleduje přenos látek v odpadech a odpadních vodách. Úniky jsou sledovány pro jednotlivé provozovny. Podnikům vzniká povinnost ohlásit všechny úniky látky (v důsledků pravidelných, nepravidelných, úmyslných, neúmyslných i havarijních činností) při překročení stanovených prahových hodnot.

Součástí hlášení je kromě výše a druhu úniku a údajů o provozovně také informace o způsobu zjištění přenosu (měření, výpočet, odhad). U odpadů je také ohlášeno, zda byly odstraněny nebo využity a údaje o tom, jaká fyzická osoba takto učinila.


\section*{Veřejně dostupná data}
IRZ je veřejný registr a proto jeho provozovatelé musí poskytnout bezplatný přístup k datům. Hlavním zdrojem informací je web \url{https://irz.cenia.cz}, který obsahuje aktuality, relevantní přůručky, vyhlášky a zákony. Je zde také přístupný vyhledávací systém, ve kterém lze vyhledávat podle různých kritérií (látky, provozovny, druhu odpadu, atd). Data však nejsou dostupná v strojově čitelné podobě (pouze jako HTML tabulka). Data jsou dostupná od roku 2004 do roku 2017.

Na webu jsou také dostupné souhrnné zprávy zpracované Životním prostředím, avšak nejnovější zpráva je z roku 2010.

Další relevantní webovou stránkou je \url{https://www.irz.cz}, avšak podle všeho není tato stránka od roku 2010 aktualizována.


\section*{Propojená data}
Data z IRZ jsou také dostupná jako data-sety na portálu \url{opendata.cz}. Jeden data-set obsahuje údaje o ohlašovaných látkách\cite{dataset-latky}. Jednotlivá hlášení jsou pak obsažena ve zvláštním data-setu\cite{dataset-hlaseni}. Data-set není zcela aktuální a obsahuje pouze data od roku 2004 do 2012.

Data-sety lze buď procházet pomocí on-line SPARQL query editoru, stáhnout jako textový soubor nebo pomocí SPARQL end-pointu. Pro účely této práce byla použita kombinace všech tří přístupů.


\section*{Zpracování}

\subsection*{Předběžný průzkum dat}
Před vypracováním práce byl nejdříve proveden průzkum dostupných propojených dat. Protože k data-setům nejsou vypracované žádné podrobné popisy ani příklady získání dat, byla struktura vyhodnocena stažením celého data-setu (textový soubor o cca 34MB) a ručním procházením.

Nakonec byly pro účely vizualizace zvoleny následující údaje o látkách:

	\begin{itemize}
		\item Název
		\item Vzorec (pokud byl dostupný)
		\item S-věty (pokyny pro bezpečné nakládání s látkou)
		\item R-věty (popis rizik spojených s látkou)
	\end{itemize}
	
Pro jednotlivá měření úniků byly zvoleny tyto údaje:

	\begin{itemize}
		\item Rok hlášení
		\item Typ úniku (do vzduchu, vody, půdy, jako odpad)
		\item Výše uniku
		\item Látka
		\item Kraj ve kterém se provozovna nachází
	\end{itemize}
	
Přesto, že data-set nemá řádný popis, je orientace v datech zjednodušena využitím standardních schémat jako je slovník \emph{RDF}, \emph{schema.org} nebo \emph{SKOS}. Díky tomu bylo na první pohled zřejmé, co které pole v data-setu znamená.

Pokud by data byla dostupná jako běžná SQL databáze s vlastními poli, byla by orientace v datech bez podrobného popisu nemožná (což je běžný případ některých veřejně dostupných data-setů veřejné správy).

Jen u pole \emph{výše úniku} bylo nutné dohledat jednotku, ve které se úniky ohlašují.


\subsection*{Nástroje}
Pro co největší dostupnost a interaktivity bylo zvoleno zpracování v podobě webové aplikace. Byly použity tyto volně dostupné nástroje a knihovny:

	\begin{description}
		\item[Python] Programovací jazyk, který je rozšířený jak pro tvorbu webových aplikací tak i pro zpracován dat
		\item[Flask] Knihovna pro tvorbu webových aplikací
		\item[SPARQLWrapper] Knihovna pro zpracování SPARQL dotazů
		\item[chart.js] Front-endová knihovna pro tvorbu grafů
		\item[PythonAnywhere.com] Poskytovatel hostingu webových aplikací
	\end{description}


\subsection*{Získání dat a seskupování}
Pro různá data byl zvolen odlišný přístup. Seznam ohlašovaných látek a příslušné údaje (vzorec, S-věty, R-věty) byly uloženy přes webové rozhraní a vloženy do zdrojového kódu aplikace jako datové soubory. Tento přístup je vhodnější, protože jde o poměrně malý objem dat, která se budou zobrazovat každým uživatelem a často.

Pro získání dat o měřeních byl zvolen jiný přístup. Data jsou získávána pomocí SPARQL dotazu podle toho, jaké byly zvoleny parametry (IRI požadované látky).

	\begin{verbatim}
	select distinct ?iri, ?emission_type, ?year, ?value where
	{
	
		?iri          sch:object <<<IRI látky>>>;
		sch:additionalType ?emission_type;
		sch:startTime ?year;
		rdf:value ?value;
		sch:instrument ?instrument;
		sch:location ?location.
		
		?place        owl:sameAs ?location.
		?place        sch:address ?addr.
		?addr         sch:addressRegion ?district.
		OPTIONAL      {?iri cenia:urceniOdpadu ?waste_designation}.
	
	} order by ?region ?year
	\end{verbatim}

Zde je vidět jedna z velkých výhod použití RDF oproti běžným relačním databázím. Jednotlivé informace nejsou uloženy v jedné entitě ale jsou reprezentována několika propojenými entitami (provozovna, přesná adresa provozovny, souřadnice provozovny, jednotlivá měření pro každý rok, atd). V relační databázi by se jednalo o několik tabulek, pro které by bylo nutné provádět velké množství komplikovaných JOIN dotazů. V tomto případě je propojení dat jednoduše řešeno pomocí SPARQL.

Získaná data jsou poté seskupena podle kraje a podle roku. Seskupování probíhá na úrovni aplikace, protože SPARQL (podobně jako SQL) neumožňuje ve zkupinách získat jednotlivé odpovědi na dotaz, ale pouze agreguje určité sloupce (tj. nemůžeme jednoduše pracovat s jednotlivými "řádky").


\subsection*{Vizualizace}
Po získání a transformaci dat je možné tvořit různé vizualizace. Pro tuto práci byly zvoleny dvě - emise látky podle krajů po rocích a počet hlášení podle druhů emise.

\subsubsection*{Emise látky podle krajů po rocích}
Uživatel zvolí látku ze seznamu a agregační funkci pro hodnotu hlášení (počet, průměrné emise, medián emisí, celkové emise). Dále může zvolit jeden ze dvou typů grafu - sloupcový, kde každý sloupec v jednom roce odpovídá jednomu kraji, nebo plošný, na kterém je vyznačen poměr každého kraje z celkové výše emisí.

\subsubsection*{Počet hlášení podle druhů emise}
Uživatel zvolí látku ze seznamu a rozmezí roků, pro které má být proveden výpočet. Je zobrazen koláčový graf, na kterém jsou znázorněny počty emisí různého druhu (do ovzduší, vody, půdy, odpad).

Dále jsou na této stránce uvedeny informace o látce (pokud jsou dostupné) jako je název, vzorec, krátký popis (dynamicky načtený z Wikipedie) a seznam S a R vět.


\section*{Využití podobné aplikace}
Hlavním účelem tvorby IRZ a jiných veřejně dostupných registrů je větší informovanost a zapojení veřejnosti. Avšak pouhé zveřejnění dat není příliš přínosné, protože běžná veřejnost nemá znalosti potřebné k analýze těchto dat a proto z nich nemá žádný užitek. Také forma, ve které jsou data poskytována může být překážkou k využívání.

Tvorba přehledných, interaktivních vizualizací zvyšuje přínos pro běžné uživatele.

Pokud jsou data zpřístupněna ve strojově čitelné podobě, není možnost tvořit podobné aplikace vymezena jenom pro správce konkrétního systému ale může se zapojit i odbornější veřejnost.

Tvorba jednoduchých aplikací jako je tato pak zvýší veřejnosti, což je důležité pro dobře fungující společnost.


\section*{Hodnocení zvolených přístupů}
Hlavním přínosem využití propojených dat je jejich "samo-dokumentace", tedy snadná orientace v datech díky použití sémantických slovníků.

SPARQL také umožňuje jednoduché dotazování nad komplexními daty, které jsou rozprostřeny do mnoha entit ("relačních tabulek").

Jednou z hlavních nevýhod je větší nárok na servery. Aplikace provádí dotazy, které musí zpracovat jiný server, poté musí aplikace zpracovat data tak, aby je mohla použít. To způsobuje značné prodlevy a podepisuje se to na rychlosti celé aplikace (načítání grafů je viditelně pomalé). Pokud by byla aplikace aktivně využívána mnoha uživateli, zřejmě by došlo k přetížení serverů.

Další nevýhodou je, že (v tomto případě) nejsou RDF data-sety poskytovány oficiálně a proto nejsou aktuální.

Tvorba této aplikace neměla žádné speciální požadavky na nástroje (např. specializovaný software). Je nutná znalost dotazovacího jazyka SPARQL ale pro vývojáře, kteří ovládají běžné dotazovací jazyky nejde o překážku.

Hlavní překážkou k tvorbě podobných aplikací je tedy nedostupnost propojených dat z oficiálních zdrojů.
 

\begin{thebibliography}{1}

	\bibitem{dataset-latky} https://linked.opendata.cz/dataset/cenia-irz-chemicals
	\bibitem{dataset-hlaseni} https://linked.opendata.cz/dataset/cenia-pollution

\end{thebibliography}




\end{document}
